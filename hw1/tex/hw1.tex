\documentclass[12pt]{article}
\usepackage[utf8]{inputenc}
\usepackage{graphicx}
\usepackage{hyperref}
\setlength{\parskip}{0.8em}
\setlength{\parindent}{0pt}
\renewcommand{\thesubsection}{\thesection.\alph{subsection}}


\title{\Large\bfseries ASSIGNMENT 1: DATASET IDENTIFICATION AND
DESCRIPTION}
\author{Chenghao Wang}
\date{\today}

\begin{document}

\maketitle

\section{Team Members}

Chenghao Wang (only me)

My github repository for this course is
~\url{https://github.com/CHENGHAO-WANG/767_homework/tree/main}.

\section{Dataset Description}

Childhood Asthma Management Program (CAMP) Dataset (for teaching purposes)
(\url{https://biolincc.nhlbi.nih.gov/teaching/})

The CAMP was a clinical trial carried out in children with asthma.
It was designed to determine the long-term effects of three treatments (budesonide,
nedocromil, or placebo) on lung functions as measured by
the Forced Expiratory Volume at 1 second (FEV1) over a
5-6.5 year period. 
(\textit{The CAMP based teaching dataset was not prepared to
reproduce the findings of this study. A number of techniques were employed to
to completely anonymize the data. The variables were permuted over the set of
participants. A random sample of approximately 2/3 of the
1041 participants was selected for inclusion in this teaching dataset.})

There are a total of 695 subjects.

There are 20 measurement occasions.
(Baseline, 2 months, 4 months, 12 months, 16 months, 24 months, 28 months,
36 months, 40 months, 44 months, 48 months, 52 months, 56 months, 60 months,
64 months, 72 months, 84 months, 96 months, 108 months, 120 months.)

\section{Variables}
\subsection{Continuous outcome}
There are a few continuous outcomes in the dataset. The main continuous
outcome is post-bronchodilator FEV1 (postBD FEV1).

The unit of PostBD FEV1 is liter (L). 

To measure PostBD FEV1, spirometry was performed no
later than 15 minutes after administrating the bronchodilator.

\subsection{Discrete outcome}
Unfortunately, there is no discrete outcome in this dataset.

I will create a binary outcome called PreBD FFB,
which is 1 if the ratio of pre-bronchodilator FEV1 (preBD FEV1) to
preBD forced vital capacity (preBD FVC) is $ < 70 \% $, and 0 otherwise (1 = obstruction, 0 = normal).
An FEV1/FVC ratio $ < 70 \% $ defines an obstructive ventilatory defect \cite{ponce2023pulmonary}.

\subsection{Grouping variable}
Treatment group (TG). 

There are three three treatments (A = bud: budesonide,
B = ned: nedocromil, C = plbo: placebo).

\subsection{Other variables}
\begin{table}[ht]
\centering
\begin{tabular}{ll}
\hline
Variable & Description \\
\hline
$age\_rz$      & Age at randomization (years) \\
$gender$    & m = male, f = female \\
$ethnic$    & w = white, b = black, h = hispanic, o = other \\
$hemog$   & Hemoglobin (g/dL) \\
$wbc$  & White blood cell count (1000 cells/$\mu$L) \\
$agehome$ & Age of current home (years) \\
$anypet$ & Any pets in home (1 = yes, 2 = no) \\
$woodstove$ & Use a wood stove for heating/cooking (1 = yes, 2 = no) \\
$dehumid$  & Use a dehumidifier (1 = yes, 2 = no, 3 = unknown) \\
$parent\_smokes$  & Either parent/partner smokes in home (1 = yes, 2 = no) \\
$any\_smokes$ & Anyone (including visitors) smokes in home (1 = yes, 2 = no) \\
\hline
\end{tabular}
\caption{Other variables that may be used as covariates}
\end{table}

\section{Descriptive Summary}


\bibliographystyle{plain}
\bibliography{references}

\end{document}
