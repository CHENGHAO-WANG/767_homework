\documentclass[12pt]{article}
\usepackage[utf8]{inputenc}
\usepackage{graphicx}
\usepackage{hyperref}
\setlength{\parskip}{0.8em}
\setlength{\parindent}{0pt}
\renewcommand{\thesubsection}{\thesection.\alph{subsection}}


\title{\Large\bfseries ASSIGNMENT 1: DATASET IDENTIFICATION AND
DESCRIPTION}
\author{Chenghao Wang}
\date{\today}

\begin{document}

\maketitle

\section{Team Members}

Chenghao Wang (only me)

\section{Dataset Description}

Childhood Asthma Management Program (CAMP) Dataset (for teaching purposes)
(\url{https://biolincc.nhlbi.nih.gov/teaching/})

The CAMP was a clinical trial carried out in children with asthma.
It was designed to determine the long-term effects of three treatments (budesonide,
nedocromil, or placebo) on lung functions as measured by
the Forced Expiratory Volume at 1 second (FEV1) over a
5-6.5 year period. 
(\textit{The CAMP based teaching dataset was not prepared to
reproduce the findings of this study. A number of techniques were employed to
to completely anonymize the data. The variables were permuted over the set of
participants. A random sample of approximately 2/3 of the
1041 participants was selected for inclusion in this teaching dataset.})

There are a total of 695 subjects.

There are 20 measurement occasions.
(Baseline, 2 months, 4 months, 12 months, 16 months, 24 months, 28 months,
36 months, 40 months, 44 months, 48 months, 52 months, 56 months, 60 months,
64 months, 72 months, 84 months, 96 months, 108 months, 120 months.)

\section{Variables}
\subsection{Continuous outcome}
There are a few continuous outcomes in the dataset. The main continuous
outcome is FEV1 (Forced Expiratory Volume in 1 second).

The unit of FEV1 is liter (L). 


\subsection{Discrete outcome}
\cite{ponce2023pulmonary}




\bibliographystyle{plain}
\bibliography{references}

\end{document}
